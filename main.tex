\documentclass[12pt,a4paper,2sides]{report}

\usepackage{lib/lib} %import libary from lib.sty %
%Thêm thư viện ở dưới dòng này.


\onehalfspacing %Giãn dòng phân nửa.
%===============================
%Thêm dữ liệu vào intro.tex qua dòng lệnh.
%Addition into data at intro.tex file via command.
\newcommand{\khoa}{công nghệ thông tin} %TÊN NGÀNH, KHOA.
\newcommand{\bai}{bài}
\newcommand{\de}{tên đề tài}
\newcommand{\monhoc}{môn học} %TÊN MÔN HỌC 
\newcommand{\gvhd}{Tên giảng viên hướng dẫn} %TÊN CỦA GIẢNG VIÊN HƯỚNG DẪN
\newcommand{\tacgia}{Tên tác giả} %TÊN CỦA NGƯỜI VIẾT
\newcommand{\mstacgia}{Mã tác giả} %MÃ SỐ SINH VIÊN
\newcommand{\svhai}{}
\newcommand{\msvhai}{}
\newcommand{\svba}{}
\newcommand{\msvba}{}
\newcommand{\svtu}{}
\newcommand{\msvtu}{}
\newcommand{\nam}{2021}
%Acknowledge page (the second page in docs)
\newcommand{\loicamon}{
		Với những kiến thức em tích lũy được qua những ngày học tập, đây là kết quả của quá trình học tập của em và nhóm cùng đưa ra ý kiến và thảo luận, tổng hợp kiến thức lại với nhau.\\
}
%tabl
\begin{document}	
\pagenumbering{roman}
% \input{intro.tex}
\begin{center}
	\large{\textbf{TỔNG LIÊN ĐOÀN LAO ĐỘNG VIỆT NAM}} \\
	\large{\textbf{TRƯỜNG ĐẠI HỌC TÔN ĐỨC THẮNG}} \\
	\large{\textbf{\MakeUppercase{KHOA \khoa}}} \\\vspace*{1cm}	
	\includegraphics[width=0.5\linewidth]{lib/TDTlogo.jpg}\\\vspace*{1cm}	
	\Large{\textbf{\MakeUppercase{\bai}}}\\		
	\LARGE{\textbf{\MakeUppercase{\de}}}\\
	\Large{\textbf{\MakeUppercase{\monhoc}}}\vspace*{1.5cm}
\begin{flushright}			

	\large{\textit{Người thực hiện:}} \\
	\large{\textbf{\tacgia}}\\
	\large{\textbf{\mstacgia}}\\
	\large{\textbf{\svhai}}\\
	\large{\textbf{\msvhai}}\\
	\large{\textbf{\svba}}\\
	\large{\textbf{\msvba}}\\
	\large{\textit{Giảng viên hướng dẫn:}} \\
	\large{\textbf{\gvhd}} 
	\vspace*{1.5cm}
\end{flushright}
	\large{\textbf{Thành phố Hồ Chí Minh, \nam.}}
\end{center}	
	\newpage
\begin{center}
	\Large{\textbf{LỜI CẢM ƠN}}
\end{center}

	Em xin được gửi lời cảm ơn đến giảng viên \gvhd \mbox{} đã tận tình giảng dạy và giúp đỡ em trong việc hoàn thành bài tập, cũng như hiểu vấn đề của môn học đề ra.\\
	
	Với những kiến thức em tích lũy được qua những ngày học tập, đây là kết quả của quá trình học tập của em. Tuy vẫn còn nhiều mặt còn hạn chế, nhưng em sẽ cố gắng để đạt được kết quả tốt nhất có thể.\\
	\newpage
\begin{center}
	\Large{\textbf{BÀI TIỂU LUẬN ĐƯỢC HOÀN THÀNH}} \\
	\Large{\textbf{TẠI TRƯỜNG ĐẠI HỌC TÔN ĐỨC THẮNG}} \\
\end{center}	
Em xin cam đoan đây là bài báo cáo sản phẩm \bai \mbox{} của chỉ riêng em. Các nội dung nghiên cứu, kết quả trong đề tài này là trung thực và chưa được công bố dưới bất kỳ hình thức nào. Những số liệu trong các bảng biểu phục vụ cho việc phân tích, nhận xét, đánh giá được chính tác giả thu thập từ các nguồn khác nhau có ghi rõ trong phần tài liệu tham khảo.\\

Ngoài ra, trong đề tài còn sử dụng một số nhận xét, đánh giá cũng như số liệu của các tác giả khác, cơ quan tổ chức khác đều có trích dẫn và chú thích nguồn gốc rõ ràng và cụ thể.\\

Nếu phát hiện có bất kỳ sự gian lận nào em xin hoàn toàn chịu trách nhiệm về nội dung của bài \bai. Trường đại học Tôn Đức Thắng không liên quan đến những vi phạm tác quyền trong quá trình thực hiện của em.\\

\begin{center}
	\hspace*{7cm}Trường ĐH Tôn Đức Thắng,\\
	\hspace*{7cm}Ngày ........ tháng ........ năm \nam.\\
	\hspace*{7cm}Sinh viên thực hiện,\\
	\hspace*{7cm}(Ký và ghi rõ họ tên)\\
	\vspace*{0.2cm}
	\vspace*{2cm}
	 %\includegraphics[width=0.7\linewidth]{lib/signate.png}\\
	\hspace*{7cm}\tacgia.
\end{center}		
	\newpage
\begin{center}
	\Large{\textbf{PHẦN XÁC NHẬN VÀ ĐÁNH GIÁ}}
\end{center}
	\textbf{Phần đánh giá của giảng viên chấm bài:}\\
	...............................................................................................................................................\\
	...............................................................................................................................................\\
	...............................................................................................................................................\\
	...............................................................................................................................................\\
	...............................................................................................................................................\\
	...............................................................................................................................................\\
	...............................................................................................................................................\\
	...............................................................................................................................................\\
	...............................................................................................................................................
\begin{center}
	\hspace*{5cm} TP. Hồ Chí Minh, ngày..... tháng..... năm \nam.\\
	\hspace*{5cm} Giảng viên chấm bài,\\
	\vspace*{1.2cm}
	\hspace*{5cm} .......................................
\end{center}
	\vspace*{0.5cm}
	\textbf{Phần đánh giá của giảng viên hướng dẫn:}\\
	...............................................................................................................................................\\
	...............................................................................................................................................\\
	...............................................................................................................................................\\
	...............................................................................................................................................\\
	...............................................................................................................................................\\
	...............................................................................................................................................\\
	...............................................................................................................................................\\
	...............................................................................................................................................\\
	...............................................................................................................................................
\begin{center}
	\hspace*{5cm} TP. Hồ Chí Minh, ngày..... tháng..... năm \nam.\\
	\hspace*{5cm} Giảng viên hướng dẫn,\\
	\vspace*{2cm}
	\hspace*{5cm} \gvhd.
\newpage
\end{center}

\newpage
\clearpage
%--------------Mục lục
\dominitoc
\tableofcontents %pages
%if you want to show table of contents of chapter, set \dominitoc under line \chapter{name}
%\listoffigures %pictures
\newpage
\clearpage
%\setcounter{page}{-1}
%\listoftables
%\begin{figure/tables}
%\centering
%Sourcode/Import
%\caption
%\end{fugure/tables}
\clearpage
\pagenumbering{arabic}
\setcounter{page}{1}
\clearpage
%\Control to file to create subfile.\\
%\input{part1.tex}


% \chapter{Tóm tắt \& giới thiệu}
% \minitoc
\newpage
\section{Giới thiệu đề tài} % Về tính ứng dụng của đề tài thông qua môn học
\subsection{Vấn đề cần giải quyết} %Vấn đề
\subsection{Mục tiêu} %Mong muốn của mình sau khi giải quyết vấn đề này mình thu lại được những gì?
\subsection{Giải pháp} %Các giải pháp và so sánh, tại sao lại chọn giải pháp này, các giải pháp này được hướng dẫn ở trên lớp, và chỉ tập trung vào các giải pháp mình sử dụng.
\subsection{Thành phần đóng góp} %Gồm các thành viên nhóm được phân công làm gì, và thời gian cho từng hạng mục.
\subsection{Tài nguyên hỗ trợ} % Các phần mềm, các thư viện mình sử dụng trong đồ án, bài tập này.
\section{Xây dựng mô hình} %Xây dựng từ các sơ sở lý thuyết, về các mô hình (erd,uc,...), bảng biểu
\subsection{Sơ đồ ERD}
\subsection{Sơ đồ tác nhân}
\subsection{Sơ đồ} %Có thể chọn sơ đồ tiến trình hoạt động, hoặc những thứ tương tự, thông thường ít nhấ là 3 sơ đồ.

\section{Hiện thực hoá}
% Giới thiệu về các chức năng mình làm được
\subsection{}

\section{Đánh giá}
\subsection{Ưu nhược điểm}
\subsubsection{Điểm mạnh}
\subsubsection{Điểm yếu}

\subsection{Bài học nhận được}

\subsection{Phương hướng phát triển}

% \lipsum[1-4] %Thử với đoạn text này chơi.\textsl{}


%TÀI LIỆU THAM KHẢO
% Nguyễn Văn A, 2012, Tăng trưởng bền vững, Tạp chí X, truy cập ngày 15 tháng 06 năm 2012, <http://tapchiy.com/tangtruong.dpf>.
\begin{thebibliography}{5}
	\bibitem{} Tài liệu học tập trường đại học Tôn Đức Thắng.
	\bibitem{}
	\bibitem{}
\end{thebibliography}
\end{document}
